%%
%% GAC: When Smaller Is Slower — Dimensional Misalignment in Compressed LLMs
%% Target: EuroMLSys 2026 (SIGPLAN format, 6 pages excluding references)
%%

\documentclass[sigplan,10pt,nonacm]{acmart}

%% Remove ACM-specific elements for submission
\settopmatter{printacmref=false,authorsperrow=5}
\renewcommand\footnotetextcopyrightpermission[1]{}
\pagestyle{plain}

%% Packages
\usepackage{booktabs}
\usepackage{subcaption}
\usepackage{xcolor}
\usepackage{todonotes}
\usepackage{soul}
\usepackage{graphicx}
\usepackage{tikz}
\usetikzlibrary{positioning,decorations.pathreplacing,fit,backgrounds}
\usepackage{placeins}
\usepackage{colortbl}
\usepackage{enumitem}
\usepackage{algorithm}
\usepackage{algpseudocode}
\hfuzz=0pt  % Strict: report ALL overfull boxes

%% Compact spacing
\setlength{\textfloatsep}{8pt plus 2pt minus 2pt}
\setlength{\floatsep}{8pt plus 2pt minus 2pt}
\setlength{\intextsep}{6pt plus 2pt minus 2pt}
\setlength{\abovecaptionskip}{4pt}
\setlength{\belowcaptionskip}{2pt}

%% Colors matching slides
\definecolor{cblue}{RGB}{55,131,187}
\definecolor{cred}{RGB}{211,63,73}
\definecolor{cgreen}{RGB}{56,158,92}
\definecolor{corange}{RGB}{230,159,0}

\newif\ifsubmission
\submissionfalse    
% \submissiontrue

\ifsubmission
\newcommand{\mcnote}[1]{}
\newcommand{\jhnote}[1]{}
\newcommand{\tlnote}[1]{}
\else
\newcommand{\mcnote}[1]{\todo[color=violet!40,inline]{\hl{\textbf{MC:} } #1}}
\newcommand{\jhnote}[1]{\todo[color=cyan!10!white,inline]{\hl{\textbf{JH:} }\small #1}}
\newcommand{\tlnote}[1]{\todo[color=green!15!white,inline]{\hl{\textbf{TL:} }\small #1}}
\fi

%% Title
\title{Why Smaller Is Slower?\\ Dimensional Misalignment in Compressed LLMs}

%% Authors
\author{Jihao Xin}
\affiliation{
  \institution{KAUST}
  \country{Saudi Arabia}
}

\author{Tian Lyu}
\affiliation{
  \institution{KAUST}
  \country{Saudi Arabia}
}

\author{Qilong Pan}
\affiliation{
  \institution{HUMAIN AI}
  \country{Saudi Arabia}
}

\author{Kesen Wang}
\affiliation{
  \institution{HUMAIN AI}
  \country{Saudi Arabia}
}

\author{Marco Canini}
\affiliation{
  \institution{KAUST}
  \country{Saudi Arabia}
}

\begin{abstract}
Post-training compression reduces LLM parameter counts but often produces irregular tensor dimensions that degrade GPU performance---a phenomenon we call \emph{dimensional misalignment}.
We present a full-stack analysis tracing root causes at three levels: framework (e.g., PyTorch backend dispatch), library (e.g., cuBLAS kernel selection), and hardware (e.g., Tensor Core tile and L2 Cache sector).
The key insight is that model inference becomes slower due to the resulting dimensions being incompatible with the GPU execution stack.
For example, ASVD-compressed Llama-3-8B has 15\% fewer parameters yet runs no faster than the uncompressed baseline, because 95\% of its dimensions are misaligned.
We propose \textbf{GAC} (GPU-Aligned Compression), a new compression paradigm that wraps any dimension-reducing compressor and re-selects hardware-aligned dimensions via multi-choice knapsack optimization under the same parameter budget.
On Llama-3-8B with ASVD and LLM-Pruner, GAC achieves 100\% alignment, recovering up to 1.5$\times$ speedup while preserving model quality.
\end{abstract}

\keywords{LLM Compression, GPU Optimization}

\begin{document}

\maketitle


%% ===========================================
%% 1. INTRODUCTION
%% ===========================================
\section{Introduction}
\label{sec:intro}

Large Language Models (LLMs) deliver strong capabilities, but their scale hinders deployment.
Post-training compression reduces model size and can be categorized into three families: \emph{quantization}~\cite{gptq,awq}, \emph{sparsification}~\cite{sparsegpt,wanda}, and \emph{dimension reduction}~\cite{asvd,palu,pyramidkv}.
Quantization and sparsification preserve tensor shapes and are therefore compatible with existing GPU kernels.
Dimension reduction, however, alters the tensor shapes often producing \emph{irregular} dimensions (e.g., dimension reduced from 128 to 107), which conflict with GPU execution primitives---Tensor Core tiles, cuBLAS kernels, and framework backend selection---causing a counterintuitive outcome: \emph{models with \textbf{fewer} parameters run \textbf{slower} than uncompressed ones}.
We term this phenomenon \textbf{dimensional misalignment}.

\begin{figure}[t]
\centering
\resizebox{0.85\columnwidth}{!}{%
\begin{tikzpicture}[
    dim/.style={<->, line width=0.7pt, gray!80!black},
    dlabel/.style={font=\sffamily\bfseries, fill=white, inner sep=1pt}]
  % --- Matrix X: M rows x K cols ---
  \fill[cblue!8] (0,0) rectangle (1.6,2.6);
  \draw[line width=0.8pt] (0,0) rectangle (1.6,2.6);
  \node at (0.8,1.3) {\large $\mathbf{X}$};
  \draw[dim] (-0.4,0) -- (-0.4,2.6) node[midway, dlabel] {$M$};
  \draw[dim] (0,3.0) -- (1.6,3.0) node[midway, dlabel] {$K$};
  % dot
  \node at (2.4,1.8) {\LARGE $\cdot$};
  % --- Matrix W: K rows x N cols ---
  \fill[cgreen!8] (3.2,1.0) rectangle (5.8,2.6);
  \draw[line width=0.8pt] (3.2,1.0) rectangle (5.8,2.6);
  \node at (4.5,1.8) {\large $\mathbf{W}$};
  \draw[dim] (2.9,1.0) -- (2.9,2.6) node[midway, dlabel] {$K$};
  \draw[dim] (3.2,3.0) -- (5.8,3.0) node[midway, dlabel] {$N$};
  % equals
  \node at (6.6,1.3) {\LARGE $=$};
  % --- Matrix Y: M rows x N cols ---
  \fill[corange!8] (7.4,0) rectangle (10.0,2.6);
  \draw[line width=0.8pt] (7.4,0) rectangle (10.0,2.6);
  \node at (8.7,1.3) {\large $\mathbf{Y}$};
  \draw[dim] (7.1,0) -- (7.1,2.6) node[midway, dlabel] {$M$};
  \draw[dim] (7.4,3.0) -- (10.0,3.0) node[midway, dlabel] {$N$};
\end{tikzpicture}%
}
\caption{GeMM: $Y{=}X{\cdot}W$ with $X{\in}\mathbb{R}^{M \times K}$,
$W{\in}\mathbb{R}^{K \times N}$.}
\label{fig:GeMM_dims}
\vspace{-0.2cm}
\end{figure}

To see why, consider one of the core operators in LLMs, GeMM (General Matrix Multiply) $Y{=}X{\cdot}W$ (Figure~\ref{fig:GeMM_dims}), with sequence dimension~$M$, inner dimension~$K$, and output dimension~$N$.
Three compression methods each target a different axis:
\textbf{(i)}~low-rank factorization replaces $W$ with two factors $A{\cdot}B$, introducing a reduced rank $r$ as the inner dimension of two smaller GeMMs~\citep{asvd,palu,svdllm2024};
\textbf{(ii)}~structured pruning removes neurons or attention heads, shrinking~$N$~\citep{llmpruner};
\textbf{(iii)}~token/KV eviction drops sequence entries, reducing~$M$~\citep{h2o,pyramidkv}.
All three optimize \emph{size vs.\ accuracy} without checking whether the resulting dimensions satisfy GPU alignment constraints (e.g., $d \bmod 8 = 0$).
The problem is pervasive: ASVD produces 95\% misaligned dimensions, and even LLM-Pruner---whose pruning granularity is relatively coarse---still leaves 17\% of weights misaligned (Table~\ref{tab:main_results}).
Prior hardware-aware methods~\citep{halp2021,haloc2023} treat latency as a black-box signal, tying solutions to specific CNN architectures without isolating root causes or providing parameter-budget guarantees.

We propose \textbf{GAC} (GPU-Aligned Compression), a compression \emph{paradigm} that imposes hardware alignment constraints on any dimension-reducing compressor.
GAC first identifies \emph{why} certain dimensions are slow via a full-stack analysis (\S\ref{sec:analysis}), then constrains a multi-choice knapsack optimizer to re-select aligned dimensions under the same parameter budget.
GAC is a compressor-agnostic framework that restores speed without changing the upstream compressor's design.

\noindent \textbf{Contributions.}\\
\textbf{(1)}~We identify the \emph{dimensional misalignment} problem and demonstrate its prevalence across compressors (\S\ref{sec:misalignment}).\\
\textbf{(2)}~We conduct a systematic full-stack analysis tracing root causes at the framework, library, and hardware levels (\S\ref{sec:analysis}).\\
\textbf{(3)}~We formalize dimension selection as a constrained optimization and provide a dynamic programming solver (\S\ref{sec:gac}).\\
\textbf{(4)}~Preliminary evaluation on Llama-3-8B achieves up to 1.5$\times$ speedup without sacrificing accuracy (\S\ref{sec:eval}).


%% ===========================================
%% 2. DIMENSIONAL MISALIGNMENT
%% ===========================================
\section{Motivation: Dimensional Misalignment}
\label{sec:misalignment}
\begin{figure*}[t]
  \centering
  \includegraphics[width=\textwidth]{figures/scatter_1x4_meta_llama_3_8b_instruct_r0.8.pdf}
  \caption{Llama-3-8B at $\rho=20\%$. Shape: $\circ{=}$Q Head, $\square{=}$K Head, $\triangle{=}$V Head; color: \textcolor{cgreen}{green}=8-aligned, \textcolor{cred}{red}=misaligned.}
  \label{fig:dim_scatter}
  \end{figure*}


Existing LLM compression aims to preserve accuracy under a given compression ratio $\rho$.
Formally, given the set of pretrained parameters $\mathcal{W}$, we seek a compressed set $\mathcal{W}'$ that minimizes expected loss subject to the size constraint:
\begin{equation}
\min_{\mathcal{W}'} \; \mathbb{E}_{(x,y)\sim\mathcal{D}}[\mathcal{L}(\mathcal{W}'; x, y)] \quad \text{s.t.} \quad 1-|\mathcal{W}'|/|\mathcal{W}| \geq \rho.
\label{eq:compression_target}
\end{equation}
Here $\mathcal{D}$ is the data distribution, $\mathcal{L}$ is the loss, $|\mathcal{W}'|$ and $|\mathcal{W}|$ denote total parameter counts. We denote $B = (1-\rho)|\mathcal{W}|$ as the parameter budget.

Different parameters have different compression sensitivities, e.g., early and late layers are often more critical than middle layers---so budget cannot be allocated uniformly.
Existing methods proceed in two steps.
First, for each parameter $W_i$, compute an \textbf{importance score} $s_i$ using a proxy (Table~\ref{tab:importance_scores}); a higher $s_i$ reflects higher sensitivity thus should retain more parameters in $W_i$.
Second, allocate the budget $B$ by assigning each $W_i$ a dimension $d_i$ so that more important $W_i$ receive larger $d_i$:
\begin{equation}
\{d_i\} = \arg\max_{\{d_i\}} \sum_i s_i \cdot |W_i| \quad \text{s.t.} \quad |\mathcal{W}'| \leq B, \quad d_i \geq 0.
\label{eq:budget_allocation}
\end{equation}
Here $d_i$ is the compressed dimension (e.g., inner dimension or output width, varies by compression method), $|W_i|$ is the parameter count given dimension $d_i$, $|\mathcal{W}'| = \sum_i |W_i|$ is the total parameter count of the compressed model, and $B$ is the parameter budget.
Because $s_i$ and the optimum $\{d_i\}$ are continuous, the resulting dimensions are typically \emph{irregular} (e.g., 107, 108, 109) and often violate GPU alignment requirements (e.g.\ $d \bmod 8 = 0$),
\tlnote{you use $d_i$ in this sentence before for compressed dimension, and now $d$ for model dimension}
 triggering backend fallbacks and kernel switches that erase the expected speedup from fewer parameters and FLOPs.

We demonstrate the dimensional misalignment problem with a real-world example: PaLU~\citep{palu}.
We use \textbf{8-alignment} ($d \bmod 8 = 0$) as an example of alignment constraints. When dimensions are not 8-aligned, latency can increase by up to 90\% (Figure~\ref{fig:sdpa_latency}); we detail this analysis in \S\ref{sec:analysis}.
Figure~\ref{fig:palu_dist} shows that a large fraction of layers end up misaligned (e.g., 78\% in this setup).
For generality, we summarize the mainstream importance score proxies into four categories (Table~\ref{tab:importance_scores}).
We empirically measure unconstrained dimension allocation on Llama-3-8B at $\rho{=}20\%$ with all four proxies; Figure~\ref{fig:dim_scatter} demonstrates that misalignment consistently occurs across methods.  Appendix~\ref{app:scatter_ratios} shows the misalignment persists across compression ratios $\rho \in [10\%, 50\%]$.
\begin{figure}[t]
\centering
\includegraphics[width=\columnwidth]{figures/fig3_palu_dist.png}
\caption{Llama-3-8B after PaLU compression at $\rho=20\%$.}
\label{fig:palu_dist}
\end{figure}

\begin{table}[t]
\centering
\caption{Mainstream importance scores for budget allocation.}
\label{tab:importance_scores}
\small
\setlength{\tabcolsep}{2.5pt}
\begin{tabular}{@{}llll@{}}
\toprule
\textbf{Method} & \textbf{Score} & \textbf{Intuition} & \textbf{Works} \\
\midrule
Magnitude & $\|W_i\|_F$ & Weight norm & SVD-LLM~\citep{svdllm2024} \\
Activation & $\|X_i\|_F$ & Input mag.\  & ASVD~\citep{asvd} \\
Gradient & $\left|\frac{\partial \mathcal{L}}{\partial h_i} \cdot h_i\right|$ & Taylor exp.\  & \citep{taylor_pruning} \\
Fisher & $\mathrm{tr}(\mathbf{F}_i)$ & Loss curvature & PaLU \citep{palu} \\
\bottomrule
\end{tabular}
\end{table}


%% ===========================================
%% 3. FULL-STACK ANALYSIS
%% ===========================================

\section{Full-Stack Analysis}
\label{sec:analysis}
We analyze where and why dimensional misalignment causes slowdowns on an NVIDIA A100-80GB with PyTorch 2.9.1, CUDA 12.8, FP16.
Latency is measured with CUDA events (50 warmup, 200 measurement iterations, 3 trials).
Root causes fall into three layers (Figure~\ref{fig:fullstack_overview}): Framework, Library, and Hardware. We detail each layer in the following subsections.
\begin{figure}[t]
  \centering
  \begin{tikzpicture}[
      box/.style={draw=gray!60, rounded corners=5pt, minimum width=1.8cm, minimum height=1.2cm,
                  line width=0.8pt, font=\sffamily\footnotesize, align=center, inner sep=3pt},
      >=stealth, arrow/.style={->, line width=1.2pt, gray!55},
      lbl/.style={font=\sffamily\scriptsize, text=gray!40!black, above=1pt}
    ]
    \node[box, fill=cblue!18] (fw) at (0,0) {Framework\\[1pt]\scriptsize\color{gray!60!black}FlashAttention\\\scriptsize\color{gray!60!black}Math};
    \node[box, fill=corange!18] (lib) at (2.8,0) {Library\\[1pt]\scriptsize\color{gray!60!black}cuBLAS\\\scriptsize\color{gray!60!black}CUTLASS};
    \node[box, fill=cred!18] (hw) at (5.6,0) {Hardware\\[1pt]\scriptsize\color{gray!60!black}Tensor Core\\\scriptsize\color{gray!60!black}CUDA Core};
    \draw[arrow] (fw) -- (lib) node[midway, lbl] {Dispatch};
    \draw[arrow] (lib) -- (hw) node[midway, lbl] {Execute};
  \end{tikzpicture}
  \caption{PyTorch SDPA execution stack.}
  \label{fig:fullstack_overview}
  \end{figure}

\subsection{Framework Layer}
\label{sec:framework}
Existing DL frameworks such as PyTorch and TensorFlow dispatch an operation to different backends.
For example, a matrix multiply \texttt{A@B} in PyTorch may run via cuBLAS or via a Triton kernel depending on the shape and hardware, where the dispatching mechanism is hidden from the user.
In this section, we exemplify this issue via PyTorch's \textbf{SDPA} (Scaled Dot-Product Attention), the core attention mechanism:
\begin{equation}
\mathrm{Attention}(Q,K,V) = \mathrm{softmax}(Q K^\top / \sqrt{d_k}) V
\end{equation}
where $Q$, $K$, $V$ are the query, key, and value matrices and $d_k$ is the per-head dimension.
When one calls the SDPA API\footnote{\texttt{torch.nn.functional.scaled\_dot\_product\_attention(Q, K, V)}} , PyTorch may select an optimized implementation (e.g., FlashAttention) or fall back to a naive eager implementation (the ``Math'' backend).

We measured SDPA latency with inputs $Q,K,V$ of shape $(B, S, H, d)$: batch $B{=}4$, sequence length $S{=}2048$, number of heads $H{=}32$, and we sweep the per-head dimension $d$ from 64 to 256.
We observed a staircase pattern (Figure~\ref{fig:sdpa_latency}).
First, multiples of 8 are faster: e.g., $d{=}129$ incurs $\sim$90\% higher latency than $d{=}128$.
Profiling shows that PyTorch uses FlashAttention only when $d \bmod 8 = 0$; otherwise it falls back to the Math kernel.
Second, among 8-aligned dimensions, latency rises in a staircase at every 32-dimension boundary.
This is caused by FlashAttention-2's template mechanism: FA2 selects the smallest template $t \geq d$ and assigns a tile shape $B_r {\times} B_c$ accordingly (Table~\ref{tab:fa2_templates}).
Crossing a template boundary (e.g., $d{=}128 \to 129$) halves $B_c$ and roughly doubles latency.
In Figure~\ref{fig:sdpa_latency}, alternating shades mark template regions; labels such as ``$t{=}96$, $B_r {\times} B_c{=}128{\times}64$'' show the active template and tile shape.
\begin{figure}[t]
  \centering
  \includegraphics[width=\columnwidth]{figures/fig2_sdpa_latency.pdf}
  \caption{PyTorch SDPA latency across dimensions.}
  \label{fig:sdpa_latency}
  \end{figure}

\begin{table}[t]
\centering
\caption{FA2 template tiers and performance ($B{=}4$, $S{=}2048$, $H{=}32$).}
\label{tab:fa2_templates}
\small
\setlength{\tabcolsep}{3pt}
\begin{tabular}{@{}llrrr@{}}
\toprule
Region & Template & $B_r \times B_c$ & Latency & vs.\ $t{=}64$ \\
\midrule
$d{=}64$ & 64 & 128$\times$128 & 0.74\,ms & 1.0$\times$ \\
$d \in (64,96]$ & 96 & 128$\times$64 & 1.12\,ms & 1.5$\times$ \\
$d \in (96,128]$ & 128 & 128$\times$64 & 1.47\,ms & 2.0$\times$ \\
$d \in (128,160]$ & 160 & 128$\times$32 & 2.00\,ms & 2.7$\times$ \\
$d \in (160,256]$ & 192--256 & 128$\times$32 & 2.3--2.9\,ms & 3--4$\times$ \\
\bottomrule
\end{tabular}
\end{table}

\subsection{Library Layer}
\label{sec:library}

Linear algebra is dispatched to libraries (e.g., cuBLAS), where the same GeMM can be served by different kernels depending on dimensions.
We examine this via GeMM $C {=} A {\cdot} B$ with $A \in \mathbb{R}^{M \times K}$ and $B \in \mathbb{R}^{K \times N}$.
We measured GeMM latency with two of $(M, N, K)$ fixed at typical LLM sizes ($M{=}N{=}2048$, $K{=}128$) and the third dimension swept from 50\% to 100\%.
Figure~\ref{fig:GeMM_alignment} shows the results.
First, $K$ and $N$ exhibit a clear alignment effect: when the swept dimension satisfies $d \bmod 8 = 0$, latency is lower (e.g., $K$ aligned $\sim$20\,$\mu$s vs.\ misaligned 22--26\,$\mu$s, up to 30\% penalty).
Second, $M$ and $N$ show \emph{kernel-switching cliffs}: at certain boundaries (e.g., $M{=}1728 \to 1729$), latency jumps (e.g., $\sim$30\%).
We profiled with Nsight Compute to explain this: when $d \bmod 8 = 0$, cuBLAS invokes its native optimized kernel; otherwise it uses a CUTLASS-generated kernel, which is further divided into align2 (fetching 2 elements at a time) or align1.
Table~\ref{tab:cublas_tiers} summarizes the three tiers.
GeMV ($M{=}1$) exhibits a similar but smaller penalty (${\sim}$12\% on $K$, ${\sim}$4\% on $N$; Figure~\ref{fig:gemv_alignment}), consistent with GeMV being memory-bound rather than compute-bound.

\begin{figure}[t]
\centering
\includegraphics[width=\columnwidth]{figures/fig_gemv_alignment.pdf}
\caption{GeMV latency ($M{=}1$, stride-1 sweep near 4096).}
\label{fig:gemv_alignment}
\end{figure}
\begin{table}[t]
\centering
\caption{cuBLAS GeMM kernel tiers (Nsight Compute).}
\label{tab:cublas_tiers}
\small
\setlength{\tabcolsep}{2pt}
\resizebox{\columnwidth}{!}{%
\begin{tabular}{@{}clll@{}}
\toprule
\textbf{Tier} & \textbf{Condition} & \textbf{Kernel} & \textbf{MMA Instr.} \\
\midrule
\rowcolor{cgreen!10} 1 & $d \bmod 8 = 0$ & cuBLAS-native sm80 & \texttt{mma.m16n8k16} \\
\rowcolor{corange!10} 2 & $d \bmod 2 = 0$ & CUTLASS sm80 align2 & \texttt{mma.m16n8k16} \\
\rowcolor{cred!10} 3 & odd & CUTLASS sm75 align1 & \texttt{mma.m16n8k8} \\
\bottomrule
\end{tabular}%
}
\end{table}

\begin{figure*}[t]
\centering
\includegraphics[width=\textwidth]{figures/fig_GeMM_alignment.pdf}
\caption{GeMM latency with dimension sweep.}
\label{fig:GeMM_alignment}
\end{figure*}

\subsection{Hardware Layer}
\label{sec:hardware}
Beyond framework and library dispatch, misaligned dimensions also cause inefficiency from the hardware level.
We use Nsight Compute profiling to isolate two mechanisms.
\textbf{(1)~Tensor Core}: The A100 MMA instruction \texttt{mma.m16n8k16} processes tiles of $16{\times}8{\times}16$ fp16 elements; dimensions not divisible by these tile sizes leave partial tiles underutilized.
A throughput sweep near $K,N{=}4096$ confirms: aligned dimensions reach 160--175 TFLOPS while misaligned ones drop to 50--110 TFLOPS, with period-16 in $K$ and period-8 in $N$ matching the tile shape (Figure~\ref{fig:hw_alignment}a,b).
\textbf{(2)~Memory}: The A100 L2 Cache operates in 32-byte sectors; for FP16 this requires $K \bmod 16 = 0$ for full utilization.
Misaligned accesses show ${\sim}$2$\times$ bandwidth loss in microbenchmarks (Figure~\ref{fig:hw_alignment}c).

\begin{figure*}[t]
\centering
\includegraphics[width=\textwidth]{figures/fig_hw_alignment.pdf}
\caption{Hardware-level alignment penalties (sweep near 4096): (a,b)~Tensor Core throughput, (c)~L2 Cache bandwidth.}
\label{fig:hw_alignment}
\end{figure*}

\paragraph{Summary.} Table~\ref{tab:constraints} summarizes all constraints.
The minimum requirement across all layers is $d \bmod 8 = 0$; stricter alignment (mod~16, mod~32) yields further gains.
These penalties compound: a single misaligned dimension can trigger a backend fallback, a suboptimal kernel, and underutilized tiles simultaneously.

\begin{table}[t]
\centering
\caption{Full-stack alignment constraints summary.}
\label{tab:constraints}
\small
\setlength{\tabcolsep}{2.5pt}
\begin{tabular}{@{}lllr@{}}
\toprule
\textbf{Level} & \textbf{Mechanism} & \textbf{Constraint} & \textbf{Penalty} \\
\midrule
\rowcolor{cblue!8} Framework & SDPA backend & $d$\%$8{=}0$ & ${\sim}$90\% \\
\rowcolor{cblue!8} Framework & FA2 template & $d$\%$32{=}0$ & ${\sim}$30\% \\
\midrule
\rowcolor{corange!8} Library & cuBLAS GeMM & $K$/$N$\%$8{=}0$ & ${\sim}$90\% \\
\rowcolor{corange!8} Library & cuBLAS GeMV & $K$\%$8{=}0$ & ${\sim}$12\% \\
\midrule
\rowcolor{cred!8} Hardware & TC MMA & $K$\%$16{=}0$, $N$\%$8{=}0$ & ${\sim}$70\% \\
\rowcolor{cred!8} Hardware & L2 sectors & $K$\%$16{=}0$ & ${\sim}$50\% \\
\bottomrule
\end{tabular}
\end{table}

%% ===========================================
%% 4. GAC: ALIGNMENT-AWARE COMPRESSION
%% ===========================================
\section{GAC: GPU-Aligned Compression}
\label{sec:gac}

% fig_gac_framework.tex — GAC Framework Pipeline (TikZ)
% Usage: % fig_gac_framework.tex — GAC Framework Pipeline (TikZ)
% Usage: % fig_gac_framework.tex — GAC Framework Pipeline (TikZ)
% Usage: \input{figures/fig_gac_framework.tex}
\begin{figure*}[t]
\centering
\begin{tikzpicture}[scale=0.78, every node/.style={scale=0.78},
    >=stealth,
    % Colors
    cblue/.style={fill={rgb,255:red,55;green,131;blue,187}},
    cred/.style={fill={rgb,255:red,211;green,63;blue,73}},
    cgreen/.style={fill={rgb,255:red,56;green,158;blue,92}},
    corange/.style={fill={rgb,255:red,230;green,159;blue,0}},
    cgray/.style={fill=gray!15},
    % Box styles
    phase/.style={draw, rounded corners=5pt, minimum width=2.4cm, minimum height=1.6cm,
                  line width=0.9pt, text=white, font=\sffamily\small, align=center},
    constraint/.style={draw, rounded corners=3pt, minimum width=2.0cm, minimum height=0.7cm,
                       line width=0.7pt, font=\sffamily\scriptsize, align=center,
                       fill=yellow!12, draw=orange!60!black},
    result/.style={draw, rounded corners=5pt, minimum width=2.2cm, minimum height=1.6cm,
                   line width=1.2pt, font=\sffamily\small, align=center},
    lbl/.style={font=\sffamily\small, align=center},
    arrow/.style={->, line width=1.4pt, color=gray!70!black},
    dasharrow/.style={->, line width=1.0pt, dashed, color=gray!50},
  ]

  % ===== LEFT: Analysis (§3) =====
  \node[phase, cgray, text=black, minimum width=2.0cm, minimum height=1.2cm]
    (sdpa) at (0, 1.2) {\textbf{SDPA}\\[-1pt]{\scriptsize FA2 template}\\[-1pt]{\scriptsize backend}};
  \node[phase, cgray, text=black, minimum width=2.0cm, minimum height=1.2cm]
    (gemm) at (0, -0.5) {\textbf{GEMM}\\[-1pt]{\scriptsize kernel tier}\\[-1pt]{\scriptsize heuristic}};
  \node[phase, cgray, text=black, minimum width=2.0cm, minimum height=1.2cm]
    (hw) at (0, -2.2) {\textbf{Hardware}\\[-1pt]{\scriptsize TC / VecLoad}\\[-1pt]{\scriptsize L2 (neg.)}};

  % Brace for analysis
  \node[above=0.15cm of sdpa, font=\sffamily\bfseries\small] {\S3 Analysis};
  \draw[decorate, decoration={brace, amplitude=6pt, raise=2pt}, line width=0.8pt]
    ([xshift=1.2cm]sdpa.north east) -- ([xshift=1.2cm]hw.south east);

  % ===== CENTER: Constraints =====
  \node[constraint, minimum width=3.0cm, minimum height=2.8cm]
    (constraints) at (4.0, -0.5) {};
  \node[above=-0.1cm of constraints.north, font=\sffamily\bfseries\small] {Constraints};
  \node[font=\sffamily\scriptsize, align=left, anchor=north] at ([yshift=-0.3cm]constraints.north) {
    $d \bmod 8 = 0$\\[1pt]
    $d \leq$ FA2 template\\[1pt]
    dim\%8 kernel tier\\[1pt]
    avoid cliff dims\\[1pt]
    CTA wave quant.
  };

  % Arrows: analysis → constraints
  \draw[arrow] ([xshift=1.0cm]sdpa.east) -- ([yshift=0.9cm]constraints.west);
  \draw[arrow] ([xshift=1.0cm]gemm.east) -- (constraints.west);
  \draw[arrow] ([xshift=1.0cm]hw.east) -- ([yshift=-0.9cm]constraints.west);

  % ===== RIGHT: Two paths =====
  % Path A: GAC DP (new compression)
  \node[phase, cblue, minimum width=2.6cm, minimum height=1.4cm]
    (score) at (8.0, 1.0) {\textbf{Score}\\[-1pt]{\scriptsize Fisher info}};
  \node[phase, cblue, minimum width=2.6cm, minimum height=1.4cm]
    (dp) at (11.2, 1.0) {\textbf{DP Solve}\\[-1pt]{\scriptsize knapsack}\\[-1pt]{\scriptsize align to $a$}};
  \node[phase, cblue, minimum width=2.6cm, minimum height=1.4cm]
    (svd) at (14.4, 1.0) {\textbf{SVD}\\[-1pt]{\scriptsize aligned ranks}};

  % Path B: Dimension Repair (existing model)
  \node[phase, cgreen, minimum width=2.6cm, minimum height=1.4cm]
    (detect) at (8.0, -2.0) {\textbf{Detect}\\[-1pt]{\scriptsize misaligned $d$}};
  \node[phase, cgreen, minimum width=2.6cm, minimum height=1.4cm]
    (pad) at (11.2, -2.0) {\textbf{Zero-Pad}\\[-1pt]{\scriptsize $d \to \lceil d/a\rceil \cdot a$}};
  \node[phase, cgreen, minimum width=2.6cm, minimum height=1.4cm]
    (exact) at (14.4, -2.0) {\textbf{Bit-Exact}\\[-1pt]{\scriptsize output}};

  % Arrows within paths
  \draw[arrow, color={rgb,255:red,55;green,131;blue,187}] (score) -- (dp);
  \draw[arrow, color={rgb,255:red,55;green,131;blue,187}] (dp) -- (svd);
  \draw[arrow, color={rgb,255:red,56;green,158;blue,92}] (detect) -- (pad);
  \draw[arrow, color={rgb,255:red,56;green,158;blue,92}] (pad) -- (exact);

  % Constraints → paths
  \draw[arrow] (constraints.east) -- ++(0.8,0) |- ([xshift=-0.3cm]score.west)
    node[pos=0.25, above, font=\sffamily\scriptsize\itshape] {};
  \draw[arrow] (constraints.east) -- ++(0.8,0) |- ([xshift=-0.3cm]detect.west);

  % Constraints feeds into DP
  \draw[dasharrow, color=orange!70!black]
    ([yshift=0.4cm]constraints.east) -- ++(0.8,0) |- ([yshift=0.2cm]dp.west)
    node[pos=0.82, above, font=\sffamily\scriptsize\itshape, text=orange!70!black] {candidates};

  % Path labels
  \node[above=0.15cm of dp, font=\sffamily\bfseries\small, text={rgb,255:red,55;green,131;blue,187}]
    {Path A: New Compression (\S\ref{sec:gac})};
  \node[below=0.15cm of pad, font=\sffamily\bfseries\small, text={rgb,255:red,56;green,158;blue,92}]
    {Path B: Post-Hoc Repair (\S\ref{sec:repair})};

  % Results on the right
  \node[result, fill=blue!5, draw={rgb,255:red,55;green,131;blue,187},
        right=0.6cm of svd, minimum height=1.4cm] (resA) {
    {\small\bfseries 100\% aligned}\\[-1pt]
    {\scriptsize PPL 14.30}\\[-1pt]
    {\scriptsize W.Dev 2{,}217}
  };
  \node[result, fill=green!5, draw={rgb,255:red,56;green,158;blue,92},
        right=0.6cm of exact, minimum height=1.4cm] (resB) {
    {\small\bfseries +87\% speedup}\\[-1pt]
    {\scriptsize bit-exact}\\[-1pt]
    {\scriptsize +3.7\% mem}
  };
  \draw[arrow, color={rgb,255:red,55;green,131;blue,187}] (svd) -- (resA);
  \draw[arrow, color={rgb,255:red,56;green,158;blue,92}] (exact) -- (resB);

  % Input on the left
  \node[draw, rounded corners=3pt, fill=white, line width=0.7pt,
        font=\sffamily\scriptsize, align=center, minimum width=1.8cm]
    (input) at (-3.2, -0.5) {Pre-trained\\LLM\\+ budget $B$};
  \draw[arrow] (input) -- ([xshift=-0.3cm]gemm.west |- input);

\end{tikzpicture}
\caption{\textbf{GAC framework overview.}
Analysis (\S\ref{sec:analysis}) extracts alignment constraints from three layers (SDPA, GEMM, hardware).
These constraints drive two complementary solutions:
\emph{Path~A}---alignment-aware rank allocation via multi-choice knapsack DP for new compression;
\emph{Path~B}---zero-padding repair for already-compressed models.
Both paths produce fully-aligned dimensions with no accuracy loss (DP) or bit-exact output preservation (repair).}
\label{fig:gac_framework}
\end{figure*}

\begin{figure*}[t]
\centering
\begin{tikzpicture}[scale=0.78, every node/.style={scale=0.78},
    >=stealth,
    % Colors
    cblue/.style={fill={rgb,255:red,55;green,131;blue,187}},
    cred/.style={fill={rgb,255:red,211;green,63;blue,73}},
    cgreen/.style={fill={rgb,255:red,56;green,158;blue,92}},
    corange/.style={fill={rgb,255:red,230;green,159;blue,0}},
    cgray/.style={fill=gray!15},
    % Box styles
    phase/.style={draw, rounded corners=5pt, minimum width=2.4cm, minimum height=1.6cm,
                  line width=0.9pt, text=white, font=\sffamily\small, align=center},
    constraint/.style={draw, rounded corners=3pt, minimum width=2.0cm, minimum height=0.7cm,
                       line width=0.7pt, font=\sffamily\scriptsize, align=center,
                       fill=yellow!12, draw=orange!60!black},
    result/.style={draw, rounded corners=5pt, minimum width=2.2cm, minimum height=1.6cm,
                   line width=1.2pt, font=\sffamily\small, align=center},
    lbl/.style={font=\sffamily\small, align=center},
    arrow/.style={->, line width=1.4pt, color=gray!70!black},
    dasharrow/.style={->, line width=1.0pt, dashed, color=gray!50},
  ]

  % ===== LEFT: Analysis (§3) =====
  \node[phase, cgray, text=black, minimum width=2.0cm, minimum height=1.2cm]
    (sdpa) at (0, 1.2) {\textbf{SDPA}\\[-1pt]{\scriptsize FA2 template}\\[-1pt]{\scriptsize backend}};
  \node[phase, cgray, text=black, minimum width=2.0cm, minimum height=1.2cm]
    (gemm) at (0, -0.5) {\textbf{GEMM}\\[-1pt]{\scriptsize kernel tier}\\[-1pt]{\scriptsize heuristic}};
  \node[phase, cgray, text=black, minimum width=2.0cm, minimum height=1.2cm]
    (hw) at (0, -2.2) {\textbf{Hardware}\\[-1pt]{\scriptsize TC / VecLoad}\\[-1pt]{\scriptsize L2 (neg.)}};

  % Brace for analysis
  \node[above=0.15cm of sdpa, font=\sffamily\bfseries\small] {\S3 Analysis};
  \draw[decorate, decoration={brace, amplitude=6pt, raise=2pt}, line width=0.8pt]
    ([xshift=1.2cm]sdpa.north east) -- ([xshift=1.2cm]hw.south east);

  % ===== CENTER: Constraints =====
  \node[constraint, minimum width=3.0cm, minimum height=2.8cm]
    (constraints) at (4.0, -0.5) {};
  \node[above=-0.1cm of constraints.north, font=\sffamily\bfseries\small] {Constraints};
  \node[font=\sffamily\scriptsize, align=left, anchor=north] at ([yshift=-0.3cm]constraints.north) {
    $d \bmod 8 = 0$\\[1pt]
    $d \leq$ FA2 template\\[1pt]
    dim\%8 kernel tier\\[1pt]
    avoid cliff dims\\[1pt]
    CTA wave quant.
  };

  % Arrows: analysis → constraints
  \draw[arrow] ([xshift=1.0cm]sdpa.east) -- ([yshift=0.9cm]constraints.west);
  \draw[arrow] ([xshift=1.0cm]gemm.east) -- (constraints.west);
  \draw[arrow] ([xshift=1.0cm]hw.east) -- ([yshift=-0.9cm]constraints.west);

  % ===== RIGHT: Two paths =====
  % Path A: GAC DP (new compression)
  \node[phase, cblue, minimum width=2.6cm, minimum height=1.4cm]
    (score) at (8.0, 1.0) {\textbf{Score}\\[-1pt]{\scriptsize Fisher info}};
  \node[phase, cblue, minimum width=2.6cm, minimum height=1.4cm]
    (dp) at (11.2, 1.0) {\textbf{DP Solve}\\[-1pt]{\scriptsize knapsack}\\[-1pt]{\scriptsize align to $a$}};
  \node[phase, cblue, minimum width=2.6cm, minimum height=1.4cm]
    (svd) at (14.4, 1.0) {\textbf{SVD}\\[-1pt]{\scriptsize aligned ranks}};

  % Path B: Dimension Repair (existing model)
  \node[phase, cgreen, minimum width=2.6cm, minimum height=1.4cm]
    (detect) at (8.0, -2.0) {\textbf{Detect}\\[-1pt]{\scriptsize misaligned $d$}};
  \node[phase, cgreen, minimum width=2.6cm, minimum height=1.4cm]
    (pad) at (11.2, -2.0) {\textbf{Zero-Pad}\\[-1pt]{\scriptsize $d \to \lceil d/a\rceil \cdot a$}};
  \node[phase, cgreen, minimum width=2.6cm, minimum height=1.4cm]
    (exact) at (14.4, -2.0) {\textbf{Bit-Exact}\\[-1pt]{\scriptsize output}};

  % Arrows within paths
  \draw[arrow, color={rgb,255:red,55;green,131;blue,187}] (score) -- (dp);
  \draw[arrow, color={rgb,255:red,55;green,131;blue,187}] (dp) -- (svd);
  \draw[arrow, color={rgb,255:red,56;green,158;blue,92}] (detect) -- (pad);
  \draw[arrow, color={rgb,255:red,56;green,158;blue,92}] (pad) -- (exact);

  % Constraints → paths
  \draw[arrow] (constraints.east) -- ++(0.8,0) |- ([xshift=-0.3cm]score.west)
    node[pos=0.25, above, font=\sffamily\scriptsize\itshape] {};
  \draw[arrow] (constraints.east) -- ++(0.8,0) |- ([xshift=-0.3cm]detect.west);

  % Constraints feeds into DP
  \draw[dasharrow, color=orange!70!black]
    ([yshift=0.4cm]constraints.east) -- ++(0.8,0) |- ([yshift=0.2cm]dp.west)
    node[pos=0.82, above, font=\sffamily\scriptsize\itshape, text=orange!70!black] {candidates};

  % Path labels
  \node[above=0.15cm of dp, font=\sffamily\bfseries\small, text={rgb,255:red,55;green,131;blue,187}]
    {Path A: New Compression (\S\ref{sec:gac})};
  \node[below=0.15cm of pad, font=\sffamily\bfseries\small, text={rgb,255:red,56;green,158;blue,92}]
    {Path B: Post-Hoc Repair (\S\ref{sec:repair})};

  % Results on the right
  \node[result, fill=blue!5, draw={rgb,255:red,55;green,131;blue,187},
        right=0.6cm of svd, minimum height=1.4cm] (resA) {
    {\small\bfseries 100\% aligned}\\[-1pt]
    {\scriptsize PPL 14.30}\\[-1pt]
    {\scriptsize W.Dev 2{,}217}
  };
  \node[result, fill=green!5, draw={rgb,255:red,56;green,158;blue,92},
        right=0.6cm of exact, minimum height=1.4cm] (resB) {
    {\small\bfseries +87\% speedup}\\[-1pt]
    {\scriptsize bit-exact}\\[-1pt]
    {\scriptsize +3.7\% mem}
  };
  \draw[arrow, color={rgb,255:red,55;green,131;blue,187}] (svd) -- (resA);
  \draw[arrow, color={rgb,255:red,56;green,158;blue,92}] (exact) -- (resB);

  % Input on the left
  \node[draw, rounded corners=3pt, fill=white, line width=0.7pt,
        font=\sffamily\scriptsize, align=center, minimum width=1.8cm]
    (input) at (-3.2, -0.5) {Pre-trained\\LLM\\+ budget $B$};
  \draw[arrow] (input) -- ([xshift=-0.3cm]gemm.west |- input);

\end{tikzpicture}
\caption{\textbf{GAC framework overview.}
Analysis (\S\ref{sec:analysis}) extracts alignment constraints from three layers (SDPA, GEMM, hardware).
These constraints drive two complementary solutions:
\emph{Path~A}---alignment-aware rank allocation via multi-choice knapsack DP for new compression;
\emph{Path~B}---zero-padding repair for already-compressed models.
Both paths produce fully-aligned dimensions with no accuracy loss (DP) or bit-exact output preservation (repair).}
\label{fig:gac_framework}
\end{figure*}

\begin{figure*}[t]
\centering
\begin{tikzpicture}[scale=0.78, every node/.style={scale=0.78},
    >=stealth,
    % Colors
    cblue/.style={fill={rgb,255:red,55;green,131;blue,187}},
    cred/.style={fill={rgb,255:red,211;green,63;blue,73}},
    cgreen/.style={fill={rgb,255:red,56;green,158;blue,92}},
    corange/.style={fill={rgb,255:red,230;green,159;blue,0}},
    cgray/.style={fill=gray!15},
    % Box styles
    phase/.style={draw, rounded corners=5pt, minimum width=2.4cm, minimum height=1.6cm,
                  line width=0.9pt, text=white, font=\sffamily\small, align=center},
    constraint/.style={draw, rounded corners=3pt, minimum width=2.0cm, minimum height=0.7cm,
                       line width=0.7pt, font=\sffamily\scriptsize, align=center,
                       fill=yellow!12, draw=orange!60!black},
    result/.style={draw, rounded corners=5pt, minimum width=2.2cm, minimum height=1.6cm,
                   line width=1.2pt, font=\sffamily\small, align=center},
    lbl/.style={font=\sffamily\small, align=center},
    arrow/.style={->, line width=1.4pt, color=gray!70!black},
    dasharrow/.style={->, line width=1.0pt, dashed, color=gray!50},
  ]

  % ===== LEFT: Analysis (§3) =====
  \node[phase, cgray, text=black, minimum width=2.0cm, minimum height=1.2cm]
    (sdpa) at (0, 1.2) {\textbf{SDPA}\\[-1pt]{\scriptsize FA2 template}\\[-1pt]{\scriptsize backend}};
  \node[phase, cgray, text=black, minimum width=2.0cm, minimum height=1.2cm]
    (gemm) at (0, -0.5) {\textbf{GEMM}\\[-1pt]{\scriptsize kernel tier}\\[-1pt]{\scriptsize heuristic}};
  \node[phase, cgray, text=black, minimum width=2.0cm, minimum height=1.2cm]
    (hw) at (0, -2.2) {\textbf{Hardware}\\[-1pt]{\scriptsize TC / VecLoad}\\[-1pt]{\scriptsize L2 (neg.)}};

  % Brace for analysis
  \node[above=0.15cm of sdpa, font=\sffamily\bfseries\small] {\S3 Analysis};
  \draw[decorate, decoration={brace, amplitude=6pt, raise=2pt}, line width=0.8pt]
    ([xshift=1.2cm]sdpa.north east) -- ([xshift=1.2cm]hw.south east);

  % ===== CENTER: Constraints =====
  \node[constraint, minimum width=3.0cm, minimum height=2.8cm]
    (constraints) at (4.0, -0.5) {};
  \node[above=-0.1cm of constraints.north, font=\sffamily\bfseries\small] {Constraints};
  \node[font=\sffamily\scriptsize, align=left, anchor=north] at ([yshift=-0.3cm]constraints.north) {
    $d \bmod 8 = 0$\\[1pt]
    $d \leq$ FA2 template\\[1pt]
    dim\%8 kernel tier\\[1pt]
    avoid cliff dims\\[1pt]
    CTA wave quant.
  };

  % Arrows: analysis → constraints
  \draw[arrow] ([xshift=1.0cm]sdpa.east) -- ([yshift=0.9cm]constraints.west);
  \draw[arrow] ([xshift=1.0cm]gemm.east) -- (constraints.west);
  \draw[arrow] ([xshift=1.0cm]hw.east) -- ([yshift=-0.9cm]constraints.west);

  % ===== RIGHT: Two paths =====
  % Path A: GAC DP (new compression)
  \node[phase, cblue, minimum width=2.6cm, minimum height=1.4cm]
    (score) at (8.0, 1.0) {\textbf{Score}\\[-1pt]{\scriptsize Fisher info}};
  \node[phase, cblue, minimum width=2.6cm, minimum height=1.4cm]
    (dp) at (11.2, 1.0) {\textbf{DP Solve}\\[-1pt]{\scriptsize knapsack}\\[-1pt]{\scriptsize align to $a$}};
  \node[phase, cblue, minimum width=2.6cm, minimum height=1.4cm]
    (svd) at (14.4, 1.0) {\textbf{SVD}\\[-1pt]{\scriptsize aligned ranks}};

  % Path B: Dimension Repair (existing model)
  \node[phase, cgreen, minimum width=2.6cm, minimum height=1.4cm]
    (detect) at (8.0, -2.0) {\textbf{Detect}\\[-1pt]{\scriptsize misaligned $d$}};
  \node[phase, cgreen, minimum width=2.6cm, minimum height=1.4cm]
    (pad) at (11.2, -2.0) {\textbf{Zero-Pad}\\[-1pt]{\scriptsize $d \to \lceil d/a\rceil \cdot a$}};
  \node[phase, cgreen, minimum width=2.6cm, minimum height=1.4cm]
    (exact) at (14.4, -2.0) {\textbf{Bit-Exact}\\[-1pt]{\scriptsize output}};

  % Arrows within paths
  \draw[arrow, color={rgb,255:red,55;green,131;blue,187}] (score) -- (dp);
  \draw[arrow, color={rgb,255:red,55;green,131;blue,187}] (dp) -- (svd);
  \draw[arrow, color={rgb,255:red,56;green,158;blue,92}] (detect) -- (pad);
  \draw[arrow, color={rgb,255:red,56;green,158;blue,92}] (pad) -- (exact);

  % Constraints → paths
  \draw[arrow] (constraints.east) -- ++(0.8,0) |- ([xshift=-0.3cm]score.west)
    node[pos=0.25, above, font=\sffamily\scriptsize\itshape] {};
  \draw[arrow] (constraints.east) -- ++(0.8,0) |- ([xshift=-0.3cm]detect.west);

  % Constraints feeds into DP
  \draw[dasharrow, color=orange!70!black]
    ([yshift=0.4cm]constraints.east) -- ++(0.8,0) |- ([yshift=0.2cm]dp.west)
    node[pos=0.82, above, font=\sffamily\scriptsize\itshape, text=orange!70!black] {candidates};

  % Path labels
  \node[above=0.15cm of dp, font=\sffamily\bfseries\small, text={rgb,255:red,55;green,131;blue,187}]
    {Path A: New Compression (\S\ref{sec:gac})};
  \node[below=0.15cm of pad, font=\sffamily\bfseries\small, text={rgb,255:red,56;green,158;blue,92}]
    {Path B: Post-Hoc Repair (\S\ref{sec:repair})};

  % Results on the right
  \node[result, fill=blue!5, draw={rgb,255:red,55;green,131;blue,187},
        right=0.6cm of svd, minimum height=1.4cm] (resA) {
    {\small\bfseries 100\% aligned}\\[-1pt]
    {\scriptsize PPL 14.30}\\[-1pt]
    {\scriptsize W.Dev 2{,}217}
  };
  \node[result, fill=green!5, draw={rgb,255:red,56;green,158;blue,92},
        right=0.6cm of exact, minimum height=1.4cm] (resB) {
    {\small\bfseries +87\% speedup}\\[-1pt]
    {\scriptsize bit-exact}\\[-1pt]
    {\scriptsize +3.7\% mem}
  };
  \draw[arrow, color={rgb,255:red,55;green,131;blue,187}] (svd) -- (resA);
  \draw[arrow, color={rgb,255:red,56;green,158;blue,92}] (exact) -- (resB);

  % Input on the left
  \node[draw, rounded corners=3pt, fill=white, line width=0.7pt,
        font=\sffamily\scriptsize, align=center, minimum width=1.8cm]
    (input) at (-3.2, -0.5) {Pre-trained\\LLM\\+ budget $B$};
  \draw[arrow] (input) -- ([xshift=-0.3cm]gemm.west |- input);

\end{tikzpicture}
\caption{\textbf{GAC framework overview.}
Analysis (\S\ref{sec:analysis}) extracts alignment constraints from three layers (SDPA, GEMM, hardware).
These constraints drive two complementary solutions:
\emph{Path~A}---alignment-aware rank allocation via multi-choice knapsack DP for new compression;
\emph{Path~B}---zero-padding repair for already-compressed models.
Both paths produce fully-aligned dimensions with no accuracy loss (DP) or bit-exact output preservation (repair).}
\label{fig:gac_framework}
\end{figure*}

\begin{algorithm}[t]
  \caption{GAC algorithm.}
  \label{alg:gac}
  \small
  \begin{algorithmic}[1]
  \Require Model $\mathcal{M}$, compressor $\mathcal{F}$, budget $B$, unit $u$
  \Ensure Aligned dimensions $\{d_i\}_{i=1}^n$ with $d_i \in C_i$
  \Statex \textit{Step 1: Unconstrained Compression}
  \State $\{d_i^*\}, \{s_i\} \gets \mathcal{F}(\mathcal{M}, B)$ \Comment{misaligned dims \& scores}
  \Statex \textit{Step 2: Dimension Sweep}
  \For{each weight $i = 1, \ldots, n$}
    \State Sweep aligned dims near $d_i^*$ using heuristics constraints from \S\ref{sec:analysis}
    \State $C_i \gets$ candidate aligned dims avoiding performance cliffs
  \EndFor
  \Statex \textit{Step 3: Constrained Optimization (Knapsack DP)}
  \State $B' \gets B / u$ \Comment{quantize budget}
  \State $D[0..n][0..B'] \gets -\infty$;\; $D[0][0] \gets 0$
  \For{$i = 1$ to $n$}
    \For{each $d_{ij} \in C_i$}
      \State $v_{ij} \gets s_i \cdot (|W_i(d_{ij})| {-} |W_i^*|)$;\; $w'_{ij} \gets |W_i(d_{ij})| / u$
      \For{$b = w'_{ij}$ to $B'$}
        \State $D[i][b] \gets \max\!\bigl(D[i][b],\; D[i{-}1][b{-}w'_{ij}] + v_{ij}\bigr)$
      \EndFor
    \EndFor
  \EndFor
  \State Backtrack from $\arg\max_b D[n][b]$ to get $\{d_i\}$
  \State \Return $\{d_i\}$
  \end{algorithmic}
  \end{algorithm}
  
To bridge the gap between compression and alignment, we propose \textbf{GAC} (GPU-Aligned Compression), a paradigm that makes budget allocation system-aware so fewer parameters translate into real speedup.
GAC wraps any dimension-reducing compressor with a post-processing step, re-selecting dimensions to satisfy alignment constraint.
Given a model $\mathcal{M}$ with $n$ compressible weights $\mathcal{W}$, a compressor $\mathcal{F}$, and a parameter budget $B {=} (1{-}\rho)\,|\mathcal{W}|$ ($\rho$: compression ratio), GAC produces a fully aligned model in three steps
(Figure~\ref{fig:gac_framework}; Algorithm~\ref{alg:gac}).

\subsection{Step 1: Misaligned Compression}
\label{sec:step1}

We first apply $\mathcal{F}$ to $\mathcal{M}$ without alignment constraints.
$\mathcal{F}$ can be any established dimension-reducing compressor---e.g., ASVD~\citep{asvd} (SVD factorization) or LLM-Pruner~\citep{llmpruner} (structured pruning).
Internally, $\mathcal{F}$ computes a per-weight importance score $s_i$ using one of the proxies in Table~\ref{tab:importance_scores} (e.g., activation magnitude for ASVD, gradient-based Taylor expansion for LLM-Pruner), then allocates dimensions $\{d_i^*\}$ proportionally: higher $s_i$ retains a larger $d_i^*$.
Because $s_i$ and the resulting $\{d_i^*\}$ are continuous, the compressed model is \emph{misaligned}.

\subsection{Step 2: Dimension Sweep}
\label{sec:step2}

A na\"ive fix would round every $d_i^*$ to the nearest sweetpoint (e.g. multiple of)
\tlnote{multiple of \texttt{8}?} from the Constraints Table~\ref{tab:constraints}.
However, we cannot hard-code alignment rounding rules because different operators exhibit different behavior across different platforms (e.g., GPU architecture, PyTorch version).
A fixed heuristic that works on one platform may miss cliffs or exclude valid dimensions on another.

Instead, GAC selects candidates \emph{empirically}.
We use the heuristic constraints (e.g., $d \bmod 8 = 0$, $d \bmod 16 = 0$) to narrow the search space, then profile the kernel latency at each candidate near $d_i^*$ to verify it avoids performance cliffs on the \emph{actual} platform.
This produces a candidate set $C_i$.
For example, given $d_i^*{=}107.3$, the sweep yields $C_i{=}\{96, 104, 112, 128\}$: dimension 107 is excluded because it triggers a cuBLAS Tier-3 kernel (\S\ref{sec:library}), while 104 and 112 both land in Tier-1.
Because the sweep is hardware-specific, GAC adapts to different GPU architectures without manual tuning.

\subsection{Step 3: Constrained Optimization}
\label{sec:step3}

With misaligned dimensions $\{d_i^*\}$, importance scores $\{s_i\}$, and candidate sets $\{C_i\}$ in hand, we now select one aligned dimension per weight under the parameter budget.
Na\"ive rounding (e.g., round each $d_i^*$ to the nearest candidate) ignores two factors: (1)~different weights have different sensitivities, and (2)~rounding up at one weight consumes budget that could be spent elsewhere.
We therefore formulate a \emph{multi-choice knapsack} problem:
\begin{equation}
\max_{\{d_i\}} \sum_{i=1}^{n} s_i \cdot (|W_i| - |W_i^*|) \;\;\text{s.t.}\;\; \sum_{i} |W_i| \leq B,\;\; d_i \in C_i
\label{eq:gac}
\end{equation}
where $|W_i|$ is the parameter count of weight $W_i$ at dimension $d_i$, $|W_i^*|$ at the misaligned dimension $d_i^*$, and $C_i$ is the candidate set from Step~2.
The objective is \emph{asymmetric}: rounding up ($d_i > d_i^*$) preserves information (positive value), rounding down loses it (negative value), each scaled by the per-parameter importance $s_i$.
This lets high-importance weights receive more parameters while low-importance weights absorb the cost.

We solve Eq.~\ref{eq:gac} via dynamic programming.
For each candidate $d_{ij} \in C_i$, define value $v_{ij} = s_i \cdot (|W_i(d_{ij})| - |W_i^*|)$ and cost $w_{ij} = |W_i(d_{ij})|$.
The recurrence is:
\[
D[i][b] = \max_{j} \left\{ D[i{-}1][b - w_{ij}] + v_{ij} \right\}
\]
with complexity $O(n \cdot |C_{\max}| \cdot B')$, where $n$ is the number of weight matrices, $|C_{\max}| = \max_i |C_i|$ the largest candidate set, and $B'$ the quantized budget.

\noindent \textbf{Budget quantization.}
Na\"ively, the DP table has $B$ entries equal to the raw parameter budget, which can reach billions (e.g., $0.8 \times 100 \times 1024^2 \approx 10^8$ for 100 matrices of size $1024{\times}1024$ at $\rho{=}20\%$).
However, dimension reduction has a \emph{minimum cost unit}: pruning one column of a $[M,N]$ matrix by 1 adds or removes $M$ parameters.
If we further constrain dimensions to multiples of~8, the minimum unit becomes $u = 8 \cdot M_{\min}$, where $M_{\min}$ is the smallest row count across all weights.
We quantize costs and budget by $u$: $w'_{ij} = w_{ij} / u$, $B' = B / u$.
This reduces the DP table size dramatically---in the example above, $u{=}8{\times}1024{=}8192$ shrinks the table from ${\sim}10^8$ to ${\sim}12{,}500$ entries, a reduction of $8000{\times}$.
In practice, the DP runs in under one second on CPU---negligible compared to the compression itself.



%% ===========================================
%% 5. EVALUATION
%% ===========================================
\section{Evaluation}
\label{sec:eval}

\subsection{Setup}

We evaluate on Llama-3-8B~\citep{llama3} with $\rho=15\%$ using an NVIDIA A100-80GB with PyTorch 2.9.1, CUDA 12.8, FP16.
We select two representative compressors that alter tensor dimensions in orthogonal ways:
\textbf{(1)~ASVD}~\citep{asvd}: activation-aware SVD ($W \to A \cdot B$) across all projection weights\footnote{All 32 layers $\times$ 7 projections (Q, K, V, O, gate, up, down) = 224 weights.}.
\textbf{(2)~LLM-Pruner}~\citep{llmpruner}: coupled structured pruning of MLP weights\footnote{Layers 3--31 (29/32 layers); gate\_proj as pruning root, propagating to up\_proj and down\_proj.}.
We compare the no compression \emph{baseline} with \emph{Unaligned} (original compression) and \emph{GAC}.

\subsection{Implementation}

We implement GAC as a proof-of-concept by adding the DP solver on top of the existing ASVD and LLM-Pruner codebases (not yet a standalone framework; see \S\ref{sec:future}).
The dimension sweep profiles compression-sensitive kernels\footnote{In Transformers, mainly GeMM, SDPA and GeMV.} on the target GPU to build candidate sets (\S\ref{sec:step2}); the DP solver (\S\ref{sec:step3}) then selects aligned dimensions.
No model architecture changes, no runtime overhead, and no extra inference memory are required---GAC modifies only the dimension allocation before the final compression step.

\subsection{Preliminary Results}

Table~\ref{tab:main_results} summarizes the end-to-end comparison.

\begin{table}[t]
\centering
\caption{Preliminary results on Llama-3-8B ($\rho{=}15\%$). Measured with batch${=}1$, sequence length ${=}1024$.}
\label{tab:main_results}
\small
\setlength{\tabcolsep}{3pt}
\begin{tabular}{@{}lcrccc@{}}
\toprule
\textbf{Method} & \textbf{Align\%} & \textbf{PPL} & \textbf{PiQA} & \textbf{HSwag} & \textbf{Lat.\,(ms)} \\
\midrule
\rowcolor{gray!10} Baseline & 100\% & 6.14 & 0.80 & 0.50 & 99.6 \\
\midrule
ASVD & 5\% & 34.7 & 0.58 & 0.28 & 100.5\,{\scriptsize\textcolor{cred}{(+1\%)}} \\
ASVD (GAC) & 100\% & 31.3 & 0.57 & 0.26 & 67.1\,{\scriptsize\textcolor{cgreen}{($-$33\%)}} \\
\midrule
Pruner & 83\% & 9.88 & 0.80 & 0.49 & 137.7\,{\scriptsize\textcolor{cred}{(+38\%)}} \\
Pruner (GAC) & 100\% & 9.87 & 0.78 & 0.47 & 88.0\,{\scriptsize\textcolor{cgreen}{($-$12\%)}} \\
\bottomrule
\end{tabular}
\end{table}

\noindent \textbf{Alignment.}
ASVD's unconstrained allocation produces 95\% misaligned dimensions, while LLM-Pruner produces 17\% misaligned dimensions (it only prunes MLP, so attention weights stay aligned).
GAC brings both to 100\%, snapping every dimension to an aligned candidate via the asymmetric DP objective (Eq.~\ref{eq:gac}).

\noindent \textbf{Accuracy.}
We report perplexity (WikiText-2) and two downstream tasks (PiQA, HellaSwag; 200 samples each).
For ASVD, GAC lowers PPL from 34.7 to 31.3 because the DP rounds sensitive layers \emph{up}; downstream scores stay comparable
(PiQA 0.58$\to$0.57, HellaSwag 0.28$\to$0.26).
For LLM-Pruner, PPL is nearly identical (9.88 vs.\ 9.87) and downstream accuracy is well preserved (PiQA 0.80$\to$0.78, HellaSwag 0.49$\to$0.47), confirming that aligned re-selection does not sacrifice quality.

\noindent \textbf{Latency.}
We measure prefill latency (batch${=}1$, $S{=}1024$; decode latency is left for future work).
Despite reducing parameters by 15\%, unaligned ASVD shows \emph{no speedup} (100.5\,ms vs.\ 99.6\,ms baseline)---the benefit is consumed by alignment overhead.
GAC restores a 1.5$\times$ speedup (67.1\,ms, $-$33\%).
For LLM-Pruner, even 83\% alignment still incurs +38\% latency; GAC eliminates the penalty, achieving 12\% speedup over the \emph{uncompressed} baseline.
The penalty grows with sequence length (Figure~\ref{fig:prefill_scaling}): from +19\% at $S{=}128$ to +38\% at $S{=}1024$, as longer sequences push GeMMs deeper into the compute-bound regime where alignment (\S\ref{sec:hardware}) dominates.

\begin{figure}[t]
\centering
\includegraphics[width=\columnwidth]{figures/fig_prefill_scaling.pdf}
\caption{Llama-3-8B latency across sequence lengths.}
\label{fig:prefill_scaling}
\end{figure}


%% ===========================================
%% 6. RELATED WORK AND DISCUSSION
%% ===========================================
\section{Related Work}
\label{sec:related}

\noindent \textbf{System-aware compression.}
Dimension-reducing compressors such as SVD factorization~\citep{asvd,svdllm2024,palu}, structured pruning~\citep{sparsegpt,wanda,llmpruner}, and KV eviction~\citep{h2o,quest,pyramidkv} optimize accuracy under a size budget but ignore how the resulting dimensions interact with GPU execution.
HALP~\citep{halp2021} and HALOC~\citep{haloc2023} incorporate hardware awareness into CNN compression, but treat latency as a \emph{black-box} signal: they optimize aggregate runtime without isolating \emph{why} certain dimensions are slow, and are tied to specific CNN architectures with no parameter-budget guarantee.
GAC instead identifies root causes at three levels (framework dispatch, kernel selection, hardware tile alignment; \S\ref{sec:analysis}) and constrains dimension selection directly, complementing any upstream compressor while guaranteeing both alignment and parameter budget.

\noindent \textbf{Serving-side mitigations.}
Serving systems handle misalignment \emph{reactively}.
FlashAttention-2 pads to the next template
(${\sim}$30\% overhead; \S\ref{sec:framework}); vLLM~\citep{vllm} rejects unsupported head dimensions.
These add overhead or break compatibility.
GAC prevents misalignment at compression time, eliminating such workarounds.

\section{Limitations and Future Work}
\label{sec:future}

\noindent \textbf{Framework automation.}
Our current implementation adds the GAC DP solver on top of the ASVD and LLM-Pruner codebases.
A fully general GAC framework---where users supply only a model name, compression ratio, and compressor, and receive a 100\%-aligned model---remains future work.

\noindent \textbf{Model coverage.}
We evaluate on a single dense model (Llama-3-8B).
Extending to more varieties (larger scales, MoE architectures, etc.) would test GAC's generality.

\noindent \textbf{Hardware diversity.}
All experiments use Ampere (A100).
Newer GPUs impose stricter alignment~\citep{nvidia_tensor_core_evolution2024,nvidia_hopper_whitepaper,flashattention3};
edge devices (e.g., DGX Spark) add further constraints.
Profiling across generations would yield a broader heuristic landscape.

\noindent \textbf{Latency coverage.}
We report prefill latency at batch${=}1$, $S{=}1024$.
Covering a wider range of batch sizes and sequence lengths, as well as autoregressive decode latency, would give a fuller performance picture.

\noindent \textbf{Serving-engine compatibility.}
Our benchmarks use vanilla HuggingFace PyTorch models without any inference-time optimization.
Validating GAC under optimized serving engines (e.g., vLLM~\citep{vllm}, TensorRT~\citep{tensorrt}) and DL compilers (e.g., TVM) is needed to confirm alignment gains carry over to production stacks.


%% ===========================================
%% REFERENCES
%% ===========================================
\clearpage
\bibliographystyle{ACM-Reference-Format}
\bibliography{references}


%% ===========================================
%% APPENDIX
%% ===========================================
\clearpage
\appendix
\onecolumn
\section{Dimensional Misalignment Persists Across Compression Ratios}
\label{app:scatter_ratios}
The misalignment problem persists across different compression ratios.
Figure~\ref{fig:scatter_ratios} shows dimension scatter plots for Llama-3-8B under unconstrained SVD allocation at four compression levels ($\rho{=}10\%$, $30\%$, $40\%$, $50\%$) using Fisher importance scores.
At every ratio, a substantial fraction of dimensions are misaligned, confirming that dimensional misalignment is inherent to importance-based rank allocation, not an artifact of aggressive compression.
\begin{figure*}[h]
\centering
\begin{subfigure}[t]{\textwidth}
\includegraphics[width=0.9\textwidth]{figures/scatter_1x4_meta_llama_3_8b_instruct_r0.5.pdf}
\caption{Llama-3-8B, $\rho=50\%$}
\end{subfigure}
\vspace{-0.2cm}
\begin{subfigure}[t]{\textwidth}
\includegraphics[width=0.9\textwidth]{figures/scatter_1x4_meta_llama_3_8b_instruct_r0.7.pdf}
\caption{Llama-3-8B, $\rho=40\%$}
\end{subfigure}
\vspace{-0.2cm}
\begin{subfigure}[t]{\textwidth}
\includegraphics[width=0.9\textwidth]{figures/scatter_1x4_meta_llama_3_8b_instruct_r0.8.pdf}
\caption{Llama-3-8B, $\rho=30\%$}
\end{subfigure}
\vspace{-0.2cm}
\begin{subfigure}[t]{\textwidth}
\includegraphics[width=0.9\textwidth]{figures/scatter_1x4_meta_llama_3_8b_instruct_r0.9.pdf}
\caption{Llama-3-8B, $\rho=10\%$}
\end{subfigure}

\caption{Compressed dimension distributions across compression ratios.}
\label{fig:scatter_ratios}
\end{figure*}

\end{document}
